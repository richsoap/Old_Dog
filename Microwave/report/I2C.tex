\documentclass{article}
% TITLE PAGE CONTENT %%%%%%%%%%%%%%%%%%%%%%%%
% Remember to fill this section out for each
% lab write-up.
%%%%%%%%%%%%%%%%%%%%%%%%%%%%%%%%%%%%%%%%%%%%%
\usepackage{CJK}
\usepackage{float}
\usepackage{subfig}
\usepackage{graphicx}
\usepackage{listings} % For source cod
\usepackage{xcolor}
\usepackage{geometry}
\geometry{left=1.7cm,right=1.7cm,top=2.0cm,bottom=2.0cm}
\usepackage{float}
\usepackage{subfig}
% END TITLE PAGE CONTENT %%%%%%%%%%%%%%%%%%%%
\newcommand{\ifc}{$I^2C$}

\begin{document}  % START THE DOCUMENT!

\begin{CJK}{UTF8}{gkai}
\title{实验八 字符型背光液晶显示屏}
\author{杨庆龙 \\1500012956}
\date{2018.5.16}
\maketitle

\section{实验目的}
\begin{enumerate}
  \item 了解字符型液晶显示模块的使用方法
  \item 掌握并行端口模拟接口时序的方法
\end{enumerate}

\section{实验原理}
\subsection{硬件接口}
  \begin{enumerate}
    \item 1:VSS接地
    \item 2:VDD接5V电源
    \item 3:V0用于对比度调节
    \item 4:RS寄存器,高电平时选择数据,低电平时选择指令
    \item 5:RW寄存器,高电平为读操作,低电平为写操作
    \item 6:使能端
    \item 7-14:双向数据线
    \item 15-16:空引脚

  \end{enumerate}

\subsection{指令}
\begin{enumerate}
  \item 0x38:表示8位地址总线,双行显示
  \item 0x08:关闭显示
  \item 0x01:表示清屏
  \item 0x06:表示右移光标,文字不平移
  \item 0x0C:打开显示,不显示 光标
  \item 0x80:设置数据储存器的位置
  \item 0x02:光标位置复位
\end{enumerate}

\end{table}

\section{源码}

\begin{lstlisting}[language=C,numbers=left,numberstyle=\tiny,%frame=shadowbox,
   rulesepcolor=\color{red!20!green!20!blue!20},
   keywordstyle=\color{blue!70!black},
   commentstyle=\color{blue!90!},
   basicstyle=\ttfamily]
   #include <C8051F020.h>
   #include "../../includes/charlcd.h"
   #include "../../includes/time.h"
   #include "../../includes/keyboard.h"
   #include "../../includes/storage.h"
   #define STATE_PUTIN 0
   #define STATE_DONE 1
   #define CACU_ERROR 1
   #define CACU_OK 0

   unsigned char upbuffer[32];
   unsigned char upcount;
   unsigned char downbuffer[16];
   unsigned char downcount;
   unsigned char numbuf[16];
   unsigned char cacubuf[16];
   unsigned char numindex;
   unsigned char cacuindex;
   unsigned char level[] = {0,1,1,2,2};
   void add(unsigned char putin, unsigned char up) {
   	unsigned char temp;
   	switch(putin) {
   	case 'E':temp = 0x45;
   	break;
   	case 'R':temp = 0x52;
   	break;
   	case 'O':temp = 0x4f;
   	break;
   	case 0x0C:temp = 0x2B;
   	break;
   	case 0x0D:temp = 0x2D;
   	break;
   	case 0x0E:temp = 0x2A;
   	break;
   	case 0x0F:temp = 0x2F;
   	break;
   	default: temp = putin + 0x30;
   	}
   	temp = temp;
   	if(up == 1)
   		upbuffer[upcount ++] = temp;
   	else
   		downbuffer[downcount ++] = temp;
   }
   unsigned char pop() {
   	while(cacuindex > 1) {
   		if(level[cacubuf[cacuindex - 1]] >= level[cacubuf[cacuindex - 2]]) {
   			if(numindex > 1) {
   				switch(cacubuf[cacuindex-1]) {
   					case 1:numbuf[numindex - 2] = numbuf[numindex - 2] + numbuf[numindex - 1];
   					break;
   					case 2:numbuf[numindex - 2] = numbuf[numindex - 2] - numbuf[numindex - 1];
   					break;
   					case 3:numbuf[numindex - 2] = numbuf[numindex - 2] * numbuf[numindex - 1];
   					break;
   					case 4:numbuf[numindex - 2] = numbuf[numindex - 2] / numbuf[numindex - 1];
   					break;
   				}
   				numindex --;
   				cacuindex --;
   			}
   			else
   				return CACU_ERROR;
   		}
   	}
   	return CACU_OK;
   }
   void cacu() {
   	unsigned char i;
   	unsigned char tempnum = 0;
   	unsigned char result;
   	numindex = 0;
   	cacuindex = 1;
   	cacubuf[0] = 0;
   	for(i = 0;i < upcount;i ++) {
   		if(upbuffer[i] >= 0x30 && upbuffer[i] <= 0x39) {
   			tempnum *= 10;
   			tempnum = upbuffer[i] - 0x30 + tempnum;
   		}
   		else {
   			switch(upbuffer[i]) {
   			case 0x2B:numbuf[numindex++] = tempnum;
   			result = pop();
   			cacubuf[cacuindex++] = 1;
   			break;
   			case 0x2D:numbuf[numindex++] = tempnum;
   			result = pop();
   			cacubuf[cacuindex++] = 2;
   			break;
   			case 0x2A:numbuf[numindex++] = tempnum;
   			cacubuf[cacuindex++] = 3;
   			break;
   			case 0x2F:numbuf[numindex++] = tempnum;
   			cacubuf[cacuindex++] = 4;
   			}
   			tempnum = 0;
   		}
   	}
   	numbuf[numindex++] = tempnum;
   	result = pop();
   	if(result == CACU_ERROR) {
   		add('E',0);
   		add('R',0);
   		add('R',0);
   		add('O',0);
   		add('R',0);
   	}
   	else {
   		numindex = 1;
   		do {
   			numbuf[numindex ++] = numbuf[0] % 10;
   			numbuf[0] /= 10;
   		}while(numbuf[0] != 0);
   		numindex --;
   		while(numindex > 0) {
   			add(numbuf[numindex --],0);
   		}
   	}
   }


   void main(void) {
   	unsigned char nowkey;
   	unsigned char prekey;
   	unsigned char state;
   	unsigned char i;
   	unsigned char startpos;
   	unsigned char count;
   	WDTCN = 0xDE;
   	WDTCN = 0xAD;
   	sysclk_init();
   	P74OUT = 0x33;
   	lcd_init();
   	upcount = 0x00;
   	downcount = 0;
   	prekey = NOKEY;
   	state = STATE_PUTIN;
   	count = 0;
   	while(1) {
   		nowkey = getKey();
   		//count ++;
   		count %= 10;
   		if(nowkey != NOKEY && nowkey != prekey) {
   			if(state == STATE_DONE) {
   				upcount = 0;
   				downcount = 0;
   				cacubuf[0] = 0;
   				lcd_init();
   				state = STATE_PUTIN;
   			}
   			else {
   				if(nowkey == 0x0A){
   					state = STATE_DONE;
   					cacu();
   				}
   				else if(nowkey == 0x0B) {
   					upcount = 0;
   					downcount = 0;
   					cacubuf[0] = 0;
   					lcd_init();
   				}
   				else
   					add(nowkey,1);
   			}
   		}
   	 	prekey = nowkey;
   	////////////display
   	if(count == 0){
   	if(upcount < 16)
   	startpos = 0;
   	else
   	startpos = upcount - 16;
   	lcd_write_command(0x80);
   	for(i = 0;i < upcount - startpos;i ++)
   		lcd_write_data(upbuffer[i + startpos]);
   	if(startpos < 16)
   	startpos = 0;
   	else
   	startpos = downcount - 16;
   	lcd_write_command(0xC0);
   	for(i = 0;i < downcount - startpos;i ++)
   		lcd_write_data(downbuffer[i + startpos]);
   	}
   	}

   }
\end{lstlisting}
\end{CJK}

\end{document} % DONE WITH DOCUMENT!
