\documentclass{article}
% TITLE PAGE CONTENT %%%%%%%%%%%%%%%%%%%%%%%%
% Remember to fill this section out for each
% lab write-up.
%%%%%%%%%%%%%%%%%%%%%%%%%%%%%%%%%%%%%%%%%%%%%
\usepackage{CJK}
\usepackage{float}
\usepackage{subfig}
\usepackage{graphicx}
\usepackage{listings} % For source cod
\usepackage{xcolor}
\usepackage{geometry}
\geometry{left=1.7cm,right=1.7cm,top=2.0cm,bottom=2.0cm}
\usepackage{float}
\usepackage{subfig}
% END TITLE PAGE CONTENT %%%%%%%%%%%%%%%%%%%%

\begin{document}  % START THE DOCUMENT!

\begin{CJK}{UTF8}{gkai}
\title{实验二 电磁波传播特性实验}
\author{杨庆龙 \\1500012956}
\date{2018.9.27}
\maketitle

\section{实验目的}
\begin{enumerate}
  \item 掌握电磁波反射,衍射,干涉等电磁波传输特性。
\end{enumerate}

\section{实验内容}
\subsection{反射实验}
使用实验室提供的实验装置测量并记录一定射入角度下最大的反射角。又考虑到实验装置存在角度测量的系统误差,所以需要测量从两侧发射电磁波的数据,并对其取平均以消除系统误差。最终可得实验记录表\ref{t1.0}。
由于使用该实验装置测量的数据有1度的不确定性,而理论值和测量值的差并未超出1度。因此,该实验可以说明,在30度到60入射的情况下,微波基本服从反射定律。
\begin{table}[!htbp]
  \centering
  \caption{反射实验角度数据记录表}
  \label{t1.0}
\begin{tabular}{|c|c|c|c|c|c|c|c|}
  \hline
  入射角(度)&30&35&40&45&50&55&60\\
  \hline
  左侧反射角(度)&34.2&39.2&44.8&49.3&50.1&54.5&60.0\\
  \hline
  右侧反射角(度)&23.8&29.6&36.2&41.9&49.9&54.7&59.0\\
  \hline
  反射角平均值(度)&29.0&34.4&40.5&45.6&50.0&54.6&59.5\\
  \hline
\end{tabular}
\end{table}


\subsection{单缝衍射实验}
将单缝板的缝隙调节至70mm,并将其固定在实验装置上。使用微波源从单缝板的中轴线上以满足远场条件的距离向该单缝板发射波长为32mm的微波,又在单缝板后以满足远场条件的距离接收微波,并记录各个角度上微波信科的相对强度值。又由于实验装置不可避免地存在系统误差,所以需要左右两边都进行测量,将系统误差消除后,才能得到一级极小和二级极大的具体位置。
进行实验,可得数据记录表\ref{t2.0}。
\begin{table}[!htbp]
\centering
  \caption{单缝衍射实验数据记录表}
  \label{t2.0}
  \begin{tabular}{|c|c|c|c|c|c|c|c|c|c|c|}
    \hline
    角度(度)&0&3&6&9&12&15&18&21&24&27\\
    \hline
    左&82&82&82&74&63&54&42&26&10&4\\
    \hline
    右&82&82&75&62&50&44&28&11&3&4\\
    \hline
    角度(度)&30&33&36&39&42&45&48&51&54&57\\
    \hline
    左&2&0&4&4&0&6&16&12&2&2\\
    \hline
    右&0&10&10&0&10&22&14&0&6&3\\
    \hline
  \end{tabular}
\end{table}

从表格中可以得到,左右两侧的一级极小分别位于33度与30度附近,而二级极大分别位于48度和45度附近。将其取平均后可得,一级极小位于31.5度附近,而二级极大位于46.5度附近。而理论计算可得该一级极小和二级极大的角度应为27.2度与43.3度。又由于该实验所设计的步长为3度,而实验装置的分度值为1度,所以极值所在角度有高达4度的不确定性。又由于理论计算与实际测量值的差值均未超出该范围,所以,可以认为在误差允许范围内,微波服从衍射定律。

\subsection{双缝干涉实验}
将双缝衍射板调节至双缝宽均为40mm,双缝的中心距为90mm,并将其安装到实验装置上。使用微波源从单缝板的中轴线上以满足远场条件的距离向该单缝板发射波长为32mm的微波,又在单缝板后以满足远场条件的距离接收微波,并记录各个角度上微波信科的相对强度值。又由于实验装置不可避免地存在系统误差,所以需要左右两边都进行测量,将系统误差消除后,才能得到一级极小和二级极大的具体位置。进行实验,可得数据记录表\ref{t3.0}。
\begin{table}[!htbp]
  \centering
  \caption{双缝干涉实验数据记录表}
  \label{t3.0}
  \begin{tabular}{|c|c|c|c|c|c|c|c|c|c|c|}
    \hline
    角度(度)&0&3&6&9&12&15&18&21&24&27\\
    \hline
    左&80&82&52&7&13&48&60&54&26&13\\
    \hline
    右&80&40&10&30&64&72&62&39&36&44\\
    \hline
    角度(度)&30&33&36&39&42&45&48&51&54&57\\
    \hline
    左&30&33&23&27&26&26&20&6&6&10\\
    \hline
    右&44&28&47&45&36&23&4&12&20&10\\
    \hline
  \end{tabular}
\end{table}

将数据记录表\ref{t3.0}中的对应极值所在角度进行平均后,可得表\ref{t3.1}。又由于该实验具有4度的不确定度,而理论值与测量值之差均小于该值。所以,可以认为在误差允许范围内,32mm的微波也服从于衍射规律。

\begin{table}[!htbp]
  \centering
  \caption{双缝干涉实验数据记录表}
  \label{t3.0}
  \begin{tabular}{|c|c|c|c|}
    \hline
    级数&1&2\\
    \hline
    测量极大值所在位置(度)&16.5&31.5\\
    \hline
    理论极大值所在位置(度)&14&29\\
    \hline
    测量极小值所在位置(度)&25.5&34.5\\
    \hline
    理论极小值所在位置(度)&22&38\\
    \hline
  \end{tabular}
\end{table}



\end{CJK}

\end{document} % DONE WITH DOCUMENT!
